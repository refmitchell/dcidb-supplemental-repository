\documentclass[10pt, a4paper]{article}

%\usepackage{xurl}

\usepackage{abstract}

\usepackage{cite}
%\usepackage{url}
\usepackage[margin=1in]{geometry} \usepackage{multirow}
\usepackage{graphicx} \usepackage{amsmath} \usepackage{wrapfig}

\usepackage[normalem]{ulem} \useunder{\uline}{\ul}{}

\graphicspath{{img/}}

% Any configuration that should be done before the end of the preamble:
\usepackage{hyperref} \hypersetup{colorlinks=true, urlcolor=blue,
  linkcolor=blue, citecolor=blue}

\begin{document}
\title{Software Reference}
\author{Robert Mitchell: r.mitchell@ed.ac.uk}
\maketitle
\section{Introduction}
The software should be generally readable but the structure may not be
intuitive due to the development cycle. As such, we have prepared this
document to provide a quick reference for readers that want to see the
code which relates directly to equations and methods from the
paper.

\section{$\kappa$ approximation}
The $\kappa$ approximation process is spread across different stages
in different files.

\subsection{Raw $R$ estimators}
The raw $R$ estimators are computed in util/models.py. The
ReliabilityModel class provides utilities for interacting with the
light and wind reliability models and the basic estimators are fit to
the data within the constructor. The dataset itself is stored in
data/elevation\textunderscore rvals.csv and
data/wind\textunderscore rvals.csv for light elevation and wind speed
respectively.

These estimators are:

\begin{align}
  R_{Wind} &= (0.11s + 0.43)\\
  R_{Light} &=
  \begin{cases}
    (-0.07\phi + 0.80), &\text{if } \phi \leq 75^\circ,\\
    (-1.26\phi + 2.31) &\text{otherwise.}
  \end{cases}
\end{align}

Recall, these estimators produce populations with low concentrations;
to fix this we augment the estimators with small positive content.

\subsection{Constants $c_{Wind}, c_{Light}$}
These constants are included in world/cue.py; Lines 42-47, under the function
\textunderscore\textunderscore kappa\textunderscore approximation().

The final estimators are now:

\begin{align}
  R_{Wind} &= (0.11s + 0.43) + c_{Wind}\\
  R_{Light} &=
  \begin{cases}
    (-0.07\phi + 0.80) + c_{Light}, &\text{if } \phi \leq 75^\circ,\\
    (-1.26\phi + 2.31) + c_{Light} &\text{otherwise.}
  \end{cases}
\end{align}

The constants $c_{Wind}$ and $c_{Light}$ were tuned by hand.

\subsection{$\kappa$ approximation from Mardia and Jupp (2009)}
As a reminder, the $\kappa$ approximation expression is:

\begin{equation} \label{eq:kappaapprox}
\hat\kappa \approx
\begin{cases}
  2R + R^3 + \frac{5}{6}R^5, &\text{if } R < 0.53\\
  \frac{1}{2(1 - R) - (1 - R)^2 - (1-R)^3}, &\text{if } R \geq 0.85\\
  -0.4 + 1.39R + \frac{0.43}{(1-R)}, &\text{otherwise}
\end{cases}
\end{equation}

This expression can be found in code in world/cue.py; Lines 54 - 62,
within the function \textunderscore\textunderscore
kappa\textunderscore approximation().

\section{Integration models}
All integration models are available within
util/integration\textunderscore models.py. The model names have
changed over development, a lookup is provided in Table \ref{tab:lookup}.

\begin{table}[htp!]
  \centering
  \caption{\label{tab:lookup} Lookup table documenting model names which
    have changed between the paper and the code.}
  \vspace{5px}
  \begin{tabular}{|l|l|}
    \hline
    \textbf{Paper} & \textbf{Code} \\ \hline
    WVS            & CMLE          \\ \hline
    NVS            & NWS           \\ \hline
    BVS            & BWS           \\ \hline
  \end{tabular}
\end{table}


Each model is represented by a Simulator class which will have two key
functions; simulate\textunderscore treatment() and
compute\textunderscore integration(). Weights are computed (as per
each model specification) within the simulate\textunderscore
treatment() function. The final integration is computed by
compute\textunderscore integration().

\textbf{BVS Note:} As NVS is a subset of BVS (without the biases) we
used a parameter flag for BVS (bias\textunderscore window = -1) which
makes the BVS class function as the NVS model. The BVS class also
contains many references to and much logic concerning \textit{bias
  windows}; the concept was removed as one of the last stages of active
development, as such they feature prominently in code. They are not
used for any of the results presented.

\section{Evaluation process}
The evaluation process is spread across two files:
population\textunderscore generation.py and model\textunderscore
evaluation.py. As the population generation stage takes a long time to
run (especially instances where we search across two dimensions),
populations are generated and then stored in csv files for later
evaluation.

Population generation is reasonably straightforward: a simulator is
created for each parameterisation and then a Treatment (see
util/treatment.py) is created for each of the behavioural under which
the beetles were tested. 1,000,000 simulated beetles are tested on
each simlator and the results are then collected into $5^\circ$
bins. This forms the probability mass function which works as the
basis for the evaluation procedure.

The evaluation procedure reads in both the set of p.m.f.s (each
simulator will produce its own) and the behavioural data, then
evaluates the likelihood of the behavioural data against the
p.m.f. for each simulator. The likelihood results are stored in
dataframes and saved to csv files; our results are in the
results\textunderscore dfs directory.

Within model\textunderscore evaluation.py, there are functions for
evaluating each case we looked at (some of which did not make the
final paper for various reasons). Each of these follows the same
structure and each uses the same probability routine (which is
reasonably well documented); an exmaple can be found on Lines 93-103
which corresponds to the equations given in the paper.

\section{Mimic-data generation}
The data generation routines can be found in data\textunderscore
production.py The mimic-data is curated such that the mean vector of
the mimic populations is arbitrarily close to the mean vector of the
behavioural populations. There are two separate routines, one for the
cue combination data, and another for the three-day experiment. Each
routine simply reads in the corresponding behavioural data which is
then used to choose a suitable population size.

\end{document}

